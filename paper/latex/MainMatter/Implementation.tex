\chapter{Detalles de Implementación y Experimentos}\label{chapter:implementation}

Se implement\'o un contrato inteligente en el lenguaje de programaci\'on Solidity. 

Existen tres roles en el sistema: 
\begin{enumerate}
    \item usuario: puede consultar la informaci\'on p\'ublica del sistema. Toda persona posee este rol.
    \item votante: puede votar por un candidato. 
    \item due\~no: es el creador del contrato y tiene acceso a sus funcionalidades administrativas. Es el \'unico capaz de otorgar el rol votante a otra persona e incluso, a s\'i mismo. 
\end{enumerate}

Funcionalidades del contrato:
\begin{itemize}
    \item registro de votantes: el due\~no puede registrar a los votantes cuando crea el contrato. Para ello, debe proveer las direcciones p\'ublicas de Ethereum de los votantes. El sistema le asigna un identificador (ID) a cada votante. Consiste en un entero no negativo menor que la cantidad de votantes registrados y se corresponde con el \'indice del votante en la lista dada por el due\~no. A lo largo de la implementaci\'on de las funcionalidades del contrato, se utiliza el ID, en lugar de la direcci\'on p\'ublica, como forma  de representaci\'on del votante. \todo{@TODO decir xq?}
    \item registro de un voto por parte del votante: un votante puede votar por un candidato. Para ello, debe proveer la direcci\'on p\'ublica de Ethereum del candidato.
    \item registro de un voto por parte del due\~no: el due\~no puede registrar un voto, aun cuando \'el no sea el votante o el candidato en cuesti\'on. Para ello, debe proveer el identificador del votante y el identificador del candidato.
    \item conteo de votos: todo usuario puede consultar en cualquier momento la informaci\'on actualizada del conteo de votos.  La respuesta a la consulta es un arreglo \lstinline{count} de enteros no negativos, donde \lstinline{count[id]} es el n\'umero de votos obtenido por el candidato cuyo identificador es \lstinline{id}. Para obtener este resultado no se realiza el proceso de desempate, por lo que puede existir m\'as de un candidato con el mayor n\'umero de votos.  \todo{@audit el sistema aki'  es +- justo xq el tipo tiene q conocer el id}
    \item consultar qui\'en es el ganador:  todo usuario puede consultar en cualquier momento qui\'en es el ganador de la votaci\'on. \todo{@audit el sistema no es justo}
    \item destruir el contrato: el due\~no puede destruir el contrato. Esto imposibilita un uso posterior de cualquiera de sus funcionalidades. 
\end{itemize}

\todo[inline]{@TODO di lo d q el usuario debe conocer el id de un candidato si kiere saber su conteo en cualkier momento}

\todo[inline]{@TODO co'mo hicit el conteo. Detalla lo d listas d adyacencia y co'mo guardaste la funci\'on f}

\todo[inline]{@TODO co'mo hicit el desempate. Habla del heap, del path compression de los ranks y del delay del ciclo pa encontrar al ganador. Complejidad temporal}

\section{Eliminando por Tiempo}
\todo[inline]{@TODO co'mo aseguras q no voten en el mismo tiempo. Habla aki' del consenso en la red y to eso}

Sea $v_1, v_2, ..., v_k \in f$ la permutaci\'on de los pares ordenados de $f$ que cumple que $v_i$ fue el $i$-\'esimo voto en registrarse, para todo $i, 1 \leq i \leq k$. A cada votante se le puede asociar una noci\'on de ``tiempo de voto'' en el que vot\'o continuaci\'on se define una funci\'on $t: V \rightarrow \{1, 2, ..., k\}$ que :

$$
t(x) = i \quad \Leftrightarrow \quad \exists i, 1 \leq i \leq k: v_i = \langle x, y \rangle, \text{ para alg\'un } y \in V 
$$

\section{Despliegue en \textit{Kaleido}}
\todo[inline]{@TODO completar lo d kaleido}
2 nodos de Quorum con mecanismo de consenso LBFT, no firmantes. Un nodo monitor del sistema, firmante. una \textit{wallet} con m\'as de 100 cuentas

los nodos: Geth/v1.10.2-stable-6c76d816(quorum-v22.4.4)/linux-amd64/go1.16.15

tol mundo tiene un gb d ram y 0.5 vcpu

todos esta'n ubicados en la zona este de usa, hosteados por aws

caso de prueba con 100 votantes, con 21 empates en el primer lugar con 30 votos. Approx 13.7kb el bytecode de creacio'n del contrato. Se compilo' con solc versio'n 0.8.17, con el optimizador habilitado, con para'metro runs = 20 (aki' habla  del promedio d los votantes)  Consumio' un 2\% de cpu de la wallet xq todos los pedidos fueron pa alla'. Los nodos mantuvieron su cpu al 4\%. El consumo normal de la wallet es d 100mb de ram y cuan2 c lanzaron los pedidos alcanzo' un pico de 129mb. los nodos mantuvieron un uso d ram d 500mb approx. El uso del disco tampoco vario' en gran medida. se obtuvo el ganador en 300ms approx desde q c hizo el pedido desde cuba hasta q se obtuvo una respuesta. El gas consumido por el getwinner fue de 92857