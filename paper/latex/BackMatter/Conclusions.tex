\begin{conclusions}
    Los objetivos se cumplieron. Se demostr\'o que la modificaci\'on de la DFS empleada cuenta correctamente los votos y que asigna un n\'umero de votos justo a cada candidato que se encuentra en un ciclo de votaci\'on. Se prob\'o formal y rigurosamente que s\'olo se obtiene un ganador en el proceso electoral. Los algoritmos dise\~nados se consideran eficientes en la teor\'ia y demostraron serlo  en la pr\'actica, al evaluarlos en una red \textit{blockchain} Quorum, en Kaleido. 

    Se tuvo en cuenta el tiempo de los votos a la hora de eliminar un candidato  en cada iteraci\'on del IRV. Si bien es un criterio sencillo de implementar, no es quiz\'as el m\'as adecuado cuando los votantes conocen con tiempo suficiente qui\'enes son los candidatos. Esto se debe a que en esas circunstancias se est\'a premiando a un candidato sobre otro por algo que tiene poco que ver con su propuesta electoral. Por otro lado, el criterio incentiva a registrar el voto lo m\'as r\'apido posible, lo cual contribuye a que  el proceso electoral culmine en poco tiempo.

    El ganador se declara siguiendo  la definici\'on de un sistema electoral de pluralidad. Esto puede ser considerado injusto en muchos  casos de uso del sistema.

    El creador (due\~no) de una instancia del contrato inteligente implementado  registra a los votantes autorizados y tiene acceso a funcionalidades administrativas. Hay demasiado poder en ese rol: puede registrar votos de terceros, inhabilitar el contrato y s\'olo es necesario su consentimiento para  decidir qu\'e votantes pueden participar. Por otro lado, cualquier usuario puede conocer qui\'en va ganando las elecciones en cualquier momento. Esto puede considerarse injusto en muchos entornos, ya que puede predisponer la decisi\'on del votante.

    % Implementar el contrato en Solidity permite que pueda ser desplegado en cualquier cliente de Ethereum. GoQuorum es uno de esos clientes que permite la realizaci\'on de transacciones privadas, las cuales pueden ser empleadas para ocultar las decisiones de los votantes. \todo{@audit va en el capt 3}
\end{conclusions}
