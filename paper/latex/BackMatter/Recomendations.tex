\begin{recomendations}
    Se recomienda analizar con detenimiento la complejidad temporal del algoritmo de desempate, ya que puede que en casos con pocos participantes sea m\'as eficiente emplear un algoritmo $\Theta(n^2)$. 
    
    Se debe considerar el empleo de un criterio de eliminaci\'on de votantes para IRV que se adapte mejor a los entornos donde los votantes conocen con tiempo suficiente qui\'enes son los candidatos.

    Se recomienda la elecci\'on del ganador siguiendo un criterio de mayor\'ia y no uno de pluralidad. 

    El contrato debe ser redise\~nado para restarle poder al due\~no cuando as\'i se desee. Por ejemplo, se puede obligar a que la creaci\'on del contrato  necesite la firma de varias personas. Por otro lado, en  entornos donde se necesite un sistema electoral justo, se debe eliminar la posibilidad de conocer qui\'en va ganando las elecciones en cualquier momento. 
\end{recomendations}
