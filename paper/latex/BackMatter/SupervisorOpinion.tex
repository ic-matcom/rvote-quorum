\begin{opinion}
Los sistemas de voto electr\'onico han cobrado inter\'es tras la aparición de las tecnolog\'ias basadas en Blockchain, pues estas garantizan seguridad y transparencia. La actualidad del tema se refleja en las cantidad de publicaci\'ones citadas al respecto en el trabajo de diploma. Muchos gobiernos y compañ\'ias trabajan en lograr sistemas de votaci\'on electr\'onicos justos. En el propio Instituto de criptograf\'ia se desarrollan sistemas de este tipo para aplicarse en diferentes escenarios. 

El contenido del trabajo desarrollado por el estudiante está en total correspondencia con la tarea planteada. Con el diseño de algoritmos de desempate y el conteo de votos en presencia de ciclos se cumpli\'o el objetivo de proyecto de diploma. La variante del algoritmo DFS para el conteo de los votos en presencia de ciclos resulta novedosa, as\'i como el empleo de las marcas de tiempo para el desempate. Adem\'as se hace un an\'alisis correcto de la valid\'es de cada uno de los algoritmos.

En la concepción del trabajo el estudiante puso en práctica los conocimientos adquiridos a lo largo de la licenciatura particularmente en diseño y an\'alisis de algoritmos y adquirió otros como los referentes a la criptografía de clave pública, asimiló el lenguaje de programación Solidity. Mostró independencia durante su desempeño, poder de síntesis, capacidad de análisis y creatividad. Fue receptivo a las indicaciones y señalamientos realizados por su dirigente.  Lleg\'o a la implementaci\'on, se realizaron pruebas unitarias a los algoritmos programados y se verificó su posible empleo en una red real de Quorum. 

El documento presentado  cumple con las normas de redacci\'on que se exigen, se estructura correctamente, las ideas se exponen con claridad, haciendo citas de manera oportuna, definiciones, empleando gr\'aficos. Los teoremas que se enuncian se demuestran siguiendo una l\'ogica coherente.


Por todo lo planteado se propone al educando la calificación de \textbf{ “Sobresaliente” (5 puntos)} 
\end{opinion}