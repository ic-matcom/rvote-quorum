\begin{resumen}
	En este trabajo de diploma se dise\~n\'o, implement\'o y evalu\'o un sistema de votaci\'on  representativa sobre una red \textit{blockchain}  Quorum.  Se emple\'o una  modificaci\'on del algoritmo DFS para contar los votos obtenidos por cada candidato. Se aplic\'o un enfoque justo en la asignaci\'on de votos a los candidatos en los ciclos de votaci\'on. Para obtener un \'unico ganador, se implement\'o una variante del M\'etodo de Desempate Instant\'aneo (IRV), el cual es utilizado en sistemas electorales de \textit{ranking}. Se tiene en cuenta  el tiempo de voto para la eliminaci\'on de candidatos en cada ronda de IRV. Se demostr\'o que es \'unico el candidato que cumple con el criterio de eliminaci\'on y, por ende, que es \'unico el ganador del proceso electoral. Este se obtiene en un tiempo total de $O(n \log n)$, donde $n$ es la cantidad de participantes en el proceso. El sistema se encuentra soportado mediante un contrato inteligente. Este fue desplegado en una red privada de Quorum en la nube y se obtuvieron buenos resultados.
\end{resumen}

\begin{abstract}
	In this diploma work, a representative voting system was designed, implemented and evaluated on a Quorum blockchain network. A modification of the DFS algorithm was used to count the votes obtained by each candidate. A fair approach was applied in the allocation of votes to the candidates in the voting cycles. In order to obtain a single winner, a variant of the Instant-Runoff Voting method (IRV) was implemented, which is used in electoral ranking systems. Voting time is taken into account for the elimination of candidates in each round of IRV. It was shown that the candidate who meets the elimination criteria is unique and, therefore, that the winner of the electoral process is unique. This is obtained in a total time of $O(n \log n)$, where $n$ is the number of participants in the process. The system is supported by a smart contract. This was deployed in a private Quorum cloud network and good results were obtained.
\end{abstract}