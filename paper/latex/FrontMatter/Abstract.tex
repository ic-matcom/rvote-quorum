\begin{resumen}
	Resultan de inter\'es los sistemas electorales en los que los participantes pueden ser a la vez votantes y candidatos. Si un candidato emite un voto, entonces transfiere con este, tambi\'en, el voto de sus electores. A esos se les puede llamar sistemas de votaci\'on representativa. Si el voto es secreto, entonces un candidato, al emitir un voto, puede formar un ciclo de votaci\'on sin saberlo. Por ejemplo, $A$ puede votar por $B$, $B$ por $C$ y $C$ por $A$. Puede resultar dif\'icil decidir cu\'antos votos otorgarle a cada uno y, m\'as a\'un, determinar un ganador. Registrar el voto en papel y realizar manualmente el conteo pueden traer consigo varios problemas, como son los votos falsos y un incorrecto escrutinio. Con el surgimiento de los sistemas digitales se mitigaron estos y otros problemas.  Son eficientes, flexibles y baratos. Pero son vulnerables si dependen de una entidad central para registrar, calcular y revisar los votos. El car\'acter distribuido e inmutable de la tecnolog\'ia \textit{blockchain} la hace id\'onea para implementar sistemas de votaci\'on. En \cite{ovn} y \cite{borda_count} se proponen implementaciones sobre Ethereum. Quorum es una implementaci\'on de Ethereum que permite realizar transacciones privadas. En los sistemas electorales de mayor\'ia, un candidato es declarado ganador si posee la mayor\'ia de los votos v\'alidos. Si ninguno cumple con estas caracter\'isticas, entonces se decide el ganador en posteriores rondas de votaci\'on o mediante m\'etodos de desempate sobre \textit{rankings}. El M\'etodo de Desempate Instant\'aneo (IRV) es empleado en sistemas de \textit{rankings}, en los cuales los votantes ordenan a los candidatos por preferencia. En este trabajo de diploma se dise\~n\'o, implement\'o y evalu\'o un sistema de votaci\'on  representativa sobre Quorum.  Se asignaron votos de manera justa a los candidatos en los ciclos de votaci\'on y se implement\'o un mecanismo de desempate empleando IRV. El sistema se encuentra soportado mediante un contrato inteligente. Este fue desplegado en una red privada de Quorum en la nube y se obtuvieron buenos resultados.
\end{resumen}

\begin{abstract}
	Of interest are electoral systems in which participants can be both voters and candidates. If a candidate casts a vote, then he transfers with it, also, the vote of his electors. Those can be called representative voting systems. If the vote is secret, then a candidate, by casting a ballot, may unknowingly form a voting cycle. For example, $A$ can vote for $B$, $B$ for $C$, and $C$ for $A$. It can be difficult to decide how many votes to give each, let alone determine a winner. Recording the vote on paper and manually counting it can cause several problems, such as false votes and incorrect counting. With the emergence of digital systems, these and other problems were mitigated. They are efficient, flexible and cheap. But they are vulnerable if they rely on a central entity to record, calculate and review the votes. The distributed and immutable nature of blockchain technology makes it ideal for implementing voting systems. \cite{ovn} and \cite{borda_count} propose implementations on Ethereum. Quorum is an implementation of Ethereum that allows for private transactions. In majoritarian electoral systems, a candidate is declared the winner if he has a majority of the valid votes. If none meet these characteristics, then the winner is decided in subsequent rounds of voting or by means of tie-breaking methods based on rankings. The Instant-runoff Voting (IRV) is used in ranking systems, in which voters order candidates by preference. In this diploma work, a representative voting system on Quorum was designed, implemented and evaluated. Votes were fairly allocated to candidates in the voting cycles and a tie-breaking mechanism using IRV was implemented. The system is supported by a smart contract. This was deployed in a private Quorum cloud network and good results were obtained.
\end{abstract}